\documentclass[12pt,letterpaper]{hmcpset}
\usepackage[margin=1in]{geometry}
\usepackage{graphicx}

\usepackage{hyperref,url}

\usepackage{lmodern}

% info for header block in upper right hand corner
\name{Gavin Yancey}
\class{Probability and Statistics}
\assignment{Homework 1}
\duedate{Tuesday, October 27}

\usepackage{epstopdf}

% use \image{my_file.png} to add a centered image %
%   to your document!                             %
\newcommand{\image}[1]{\begin{center}\includegraphics[width=0.75\textwidth]{#1}\end{center}}

\begin{document}

\problemlist{\{a, b, c, d, e, $\text{e}_2$, f, g\}}

% a %
\begin{problem}[a]

    Create a dotplot for Rainfall.

\end{problem}

\begin{solution}
	\image{{{oct27g1.1}}}
	
	This doesn't show as almost anything about the data.  By rounding each datapoint to the nearest hundred (and so effectively making a fancy-looking histogram), we can get
	
	\image{{{oct27g1.2}}}
\end{solution}
\newpage

% b %
\begin{problem}[b]

    Create stacked dotplots for seeded and unseeded.

\end{problem}

\begin{solution}
	\image{{{oct27g2.1}}}
	
	Again, this is almost useless.  Again, we can round each datapoint to the nearest hundred to get a more useful plot:
	
	\image{{{oct27g2.2}}}
\end{solution}
\newpage

% c %
\begin{problem}[c]

    Get the mean, standard deviation, and 5-number summary of the seeded and unseeded groups.

\end{problem}

\begin{solution}
	\begin{tabular}{r|c|c|}
		& Seeded & Unseeded \\\hline
		$\bar{x}$ & 441.9846 & 164.5885 \\\hline
		$\sigma_x$ & 650.7872 & 278.4264 \\\hline
		min & 4.1 & 1 \\\hline
		$Q_1$ & 92.4 & 24.4 \\\hline
		$Q_2$ & 221.6 & 44.2 \\\hline
		$Q_3$ & 430 & 163 \\\hline
		max & 2745.6 & 1202.6 \\\hline
	\end{tabular}
	
	Note: there are (according to wikipedia, \url{https://en.wikipedia.org/wiki/Quartile#Computing_methods}), several slightly-different ways to define the first and third quartile, depending on how you want to interpolate, whether you count the median as a whole datapoint or a half-datapoint, and whether you count the endpoints of the dataset as full datapoints or as half datapoints. I calculated this using MATLAB's \texttt{quantile(dataset,[0,.25,.5,.75,1])}, interpolates over the data, but counts the endpoints less; for more details, see \url{http://www.mathworks.com/help/stats/quantiles-and-percentiles.html}. As far as I can tell, this is probably slightly different from how Minitab would calculate the quartiles, which I also does some sort of interpolation, but I can't tell what exactly; for all I could find, see \url{http://support.minitab.com/en-us/minitab/17/topic-library/basic-statistics-and-graphs/graphs/graphs-that-compare-groups/boxplots/quartiles/}. 
\end{solution}
\newpage

% d %
\begin{problem}[d]

    Plot side by side boxplots for seeded and unseeded. Why do the boxplots look so bad?

\end{problem}
	\image{oct27g3}
	
	The boxplots look rather bad because the data has a skewed distribution, with a lot of high outliers.  It might look better if we took the log of the data or something similar.
\begin{solution}

\end{solution}
\newpage

% e %
\begin{problem}[e]

    We need to re-express the data. First try natural log. Now recompute the side-by-side boxplots.

\end{problem}

\begin{solution}
	\image{oct27g4}
\end{solution}
\newpage

% e_2 %
\begin{problem}[$\text{e}_2$]

     Now try to re-express the data using cube root. How is this different from the $\log$ plot?

\end{problem}

\begin{solution}
	\image{oct27g5}
	
	This has the outliers at the top, rather than the bottom.  I'm not sure which is more representative of the data; they both look pretty good.
\end{solution}
\newpage

% f %
\begin{problem}[f]

    What do you think? Does it look like seeding produced more rain?

\end{problem}

\begin{solution}
	Yes, it probably did. The plots all show ``seeded'' as generally higher.
\end{solution}
\newpage

% g %
\begin{problem}[g]

     Can you think of any reason to prefer one or the other of the re-expressing functions (natural log or cube root)?

\end{problem}

\begin{solution}
	Depending on the circumstances, one could be more or less useful, depending on which is more representative of the actual distribution of the data. 
\end{solution}

\end{document}
