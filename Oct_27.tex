\documentclass[12pt,letterpaper]{hmcpset}
\usepackage[margin=1in]{geometry}
\usepackage{graphicx}

% info for header block in upper right hand corner
\name{}
\class{Probability and Statistics}
\assignment{Homework 1}
\duedate{Tuesday, October 27}

% use \image{my_file.png} to add a centered image %
%   to your document!                             %
\newcommand{\image}[1]{\begin{center}\includegraphics[width=0.75\textwidth]{#1}\end{center}}

\begin{document}

\problemlist{\{a, b, c, d, e, $\text{e}_2$, f, g\}}

% a %
\begin{problem}[a]

    Create a dotplot for Rainfall.

\end{problem}

\begin{solution}

\end{solution}
\newpage

% b %
\begin{problem}[b]

    Create stacked dotplots for seeded and unseeded.

\end{problem}

\begin{solution}

\end{solution}
\newpage

% c %
\begin{problem}[c]

    Get the mean, standard deviation, and 5-number summary of the seeded and unseeded groups.

\end{problem}

\begin{solution}

\end{solution}
\newpage

% d %
\begin{problem}[d]

    Plot side by side boxplots for seeded and unseeded.

\end{problem}

\begin{solution}

\end{solution}
\newpage

% e %
\begin{problem}[e]

    We need to re-express the data. First try natural log.

\end{problem}

\begin{solution}

\end{solution}
\newpage

% e_2 %
\begin{problem}[$\text{e}_2$]

     Now try to re-express the data using cube root

\end{problem}

\begin{solution}

\end{solution}
\newpage

% f %
\begin{problem}[f]

    What do you think? Does it look like seeding produced more rain?

\end{problem}

\begin{solution}

\end{solution}
\newpage

% g %
\begin{problem}[g]

     Can you think of any reason to prefer one or the other of the re-expressing functions (natural log or cube root)?

\end{problem}

\begin{solution}

\end{solution}

\end{document}
